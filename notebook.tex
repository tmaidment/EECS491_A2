
% Default to the notebook output style

    


% Inherit from the specified cell style.




    
\documentclass[11pt]{article}

    
    
    \usepackage[T1]{fontenc}
    % Nicer default font (+ math font) than Computer Modern for most use cases
    \usepackage{mathpazo}

    % Basic figure setup, for now with no caption control since it's done
    % automatically by Pandoc (which extracts ![](path) syntax from Markdown).
    \usepackage{graphicx}
    % We will generate all images so they have a width \maxwidth. This means
    % that they will get their normal width if they fit onto the page, but
    % are scaled down if they would overflow the margins.
    \makeatletter
    \def\maxwidth{\ifdim\Gin@nat@width>\linewidth\linewidth
    \else\Gin@nat@width\fi}
    \makeatother
    \let\Oldincludegraphics\includegraphics
    % Set max figure width to be 80% of text width, for now hardcoded.
    \renewcommand{\includegraphics}[1]{\Oldincludegraphics[width=.8\maxwidth]{#1}}
    % Ensure that by default, figures have no caption (until we provide a
    % proper Figure object with a Caption API and a way to capture that
    % in the conversion process - todo).
    \usepackage{caption}
    \DeclareCaptionLabelFormat{nolabel}{}
    \captionsetup{labelformat=nolabel}

    \usepackage{adjustbox} % Used to constrain images to a maximum size 
    \usepackage{xcolor} % Allow colors to be defined
    \usepackage{enumerate} % Needed for markdown enumerations to work
    \usepackage{geometry} % Used to adjust the document margins
    \usepackage{amsmath} % Equations
    \usepackage{amssymb} % Equations
    \usepackage{textcomp} % defines textquotesingle
    % Hack from http://tex.stackexchange.com/a/47451/13684:
    \AtBeginDocument{%
        \def\PYZsq{\textquotesingle}% Upright quotes in Pygmentized code
    }
    \usepackage{upquote} % Upright quotes for verbatim code
    \usepackage{eurosym} % defines \euro
    \usepackage[mathletters]{ucs} % Extended unicode (utf-8) support
    \usepackage[utf8x]{inputenc} % Allow utf-8 characters in the tex document
    \usepackage{fancyvrb} % verbatim replacement that allows latex
    \usepackage{grffile} % extends the file name processing of package graphics 
                         % to support a larger range 
    % The hyperref package gives us a pdf with properly built
    % internal navigation ('pdf bookmarks' for the table of contents,
    % internal cross-reference links, web links for URLs, etc.)
    \usepackage{hyperref}
    \usepackage{longtable} % longtable support required by pandoc >1.10
    \usepackage{booktabs}  % table support for pandoc > 1.12.2
    \usepackage[inline]{enumitem} % IRkernel/repr support (it uses the enumerate* environment)
    \usepackage[normalem]{ulem} % ulem is needed to support strikethroughs (\sout)
                                % normalem makes italics be italics, not underlines
    

    
    
    % Colors for the hyperref package
    \definecolor{urlcolor}{rgb}{0,.145,.698}
    \definecolor{linkcolor}{rgb}{.71,0.21,0.01}
    \definecolor{citecolor}{rgb}{.12,.54,.11}

    % ANSI colors
    \definecolor{ansi-black}{HTML}{3E424D}
    \definecolor{ansi-black-intense}{HTML}{282C36}
    \definecolor{ansi-red}{HTML}{E75C58}
    \definecolor{ansi-red-intense}{HTML}{B22B31}
    \definecolor{ansi-green}{HTML}{00A250}
    \definecolor{ansi-green-intense}{HTML}{007427}
    \definecolor{ansi-yellow}{HTML}{DDB62B}
    \definecolor{ansi-yellow-intense}{HTML}{B27D12}
    \definecolor{ansi-blue}{HTML}{208FFB}
    \definecolor{ansi-blue-intense}{HTML}{0065CA}
    \definecolor{ansi-magenta}{HTML}{D160C4}
    \definecolor{ansi-magenta-intense}{HTML}{A03196}
    \definecolor{ansi-cyan}{HTML}{60C6C8}
    \definecolor{ansi-cyan-intense}{HTML}{258F8F}
    \definecolor{ansi-white}{HTML}{C5C1B4}
    \definecolor{ansi-white-intense}{HTML}{A1A6B2}

    % commands and environments needed by pandoc snippets
    % extracted from the output of `pandoc -s`
    \providecommand{\tightlist}{%
      \setlength{\itemsep}{0pt}\setlength{\parskip}{0pt}}
    \DefineVerbatimEnvironment{Highlighting}{Verbatim}{commandchars=\\\{\}}
    % Add ',fontsize=\small' for more characters per line
    \newenvironment{Shaded}{}{}
    \newcommand{\KeywordTok}[1]{\textcolor[rgb]{0.00,0.44,0.13}{\textbf{{#1}}}}
    \newcommand{\DataTypeTok}[1]{\textcolor[rgb]{0.56,0.13,0.00}{{#1}}}
    \newcommand{\DecValTok}[1]{\textcolor[rgb]{0.25,0.63,0.44}{{#1}}}
    \newcommand{\BaseNTok}[1]{\textcolor[rgb]{0.25,0.63,0.44}{{#1}}}
    \newcommand{\FloatTok}[1]{\textcolor[rgb]{0.25,0.63,0.44}{{#1}}}
    \newcommand{\CharTok}[1]{\textcolor[rgb]{0.25,0.44,0.63}{{#1}}}
    \newcommand{\StringTok}[1]{\textcolor[rgb]{0.25,0.44,0.63}{{#1}}}
    \newcommand{\CommentTok}[1]{\textcolor[rgb]{0.38,0.63,0.69}{\textit{{#1}}}}
    \newcommand{\OtherTok}[1]{\textcolor[rgb]{0.00,0.44,0.13}{{#1}}}
    \newcommand{\AlertTok}[1]{\textcolor[rgb]{1.00,0.00,0.00}{\textbf{{#1}}}}
    \newcommand{\FunctionTok}[1]{\textcolor[rgb]{0.02,0.16,0.49}{{#1}}}
    \newcommand{\RegionMarkerTok}[1]{{#1}}
    \newcommand{\ErrorTok}[1]{\textcolor[rgb]{1.00,0.00,0.00}{\textbf{{#1}}}}
    \newcommand{\NormalTok}[1]{{#1}}
    
    % Additional commands for more recent versions of Pandoc
    \newcommand{\ConstantTok}[1]{\textcolor[rgb]{0.53,0.00,0.00}{{#1}}}
    \newcommand{\SpecialCharTok}[1]{\textcolor[rgb]{0.25,0.44,0.63}{{#1}}}
    \newcommand{\VerbatimStringTok}[1]{\textcolor[rgb]{0.25,0.44,0.63}{{#1}}}
    \newcommand{\SpecialStringTok}[1]{\textcolor[rgb]{0.73,0.40,0.53}{{#1}}}
    \newcommand{\ImportTok}[1]{{#1}}
    \newcommand{\DocumentationTok}[1]{\textcolor[rgb]{0.73,0.13,0.13}{\textit{{#1}}}}
    \newcommand{\AnnotationTok}[1]{\textcolor[rgb]{0.38,0.63,0.69}{\textbf{\textit{{#1}}}}}
    \newcommand{\CommentVarTok}[1]{\textcolor[rgb]{0.38,0.63,0.69}{\textbf{\textit{{#1}}}}}
    \newcommand{\VariableTok}[1]{\textcolor[rgb]{0.10,0.09,0.49}{{#1}}}
    \newcommand{\ControlFlowTok}[1]{\textcolor[rgb]{0.00,0.44,0.13}{\textbf{{#1}}}}
    \newcommand{\OperatorTok}[1]{\textcolor[rgb]{0.40,0.40,0.40}{{#1}}}
    \newcommand{\BuiltInTok}[1]{{#1}}
    \newcommand{\ExtensionTok}[1]{{#1}}
    \newcommand{\PreprocessorTok}[1]{\textcolor[rgb]{0.74,0.48,0.00}{{#1}}}
    \newcommand{\AttributeTok}[1]{\textcolor[rgb]{0.49,0.56,0.16}{{#1}}}
    \newcommand{\InformationTok}[1]{\textcolor[rgb]{0.38,0.63,0.69}{\textbf{\textit{{#1}}}}}
    \newcommand{\WarningTok}[1]{\textcolor[rgb]{0.38,0.63,0.69}{\textbf{\textit{{#1}}}}}
    
    
    % Define a nice break command that doesn't care if a line doesn't already
    % exist.
    \def\br{\hspace*{\fill} \\* }
    % Math Jax compatability definitions
    \def\gt{>}
    \def\lt{<}
    % Document parameters
    \title{EECS491 - A2 - E1 - tdm47}
    
    
    

    % Pygments definitions
    
\makeatletter
\def\PY@reset{\let\PY@it=\relax \let\PY@bf=\relax%
    \let\PY@ul=\relax \let\PY@tc=\relax%
    \let\PY@bc=\relax \let\PY@ff=\relax}
\def\PY@tok#1{\csname PY@tok@#1\endcsname}
\def\PY@toks#1+{\ifx\relax#1\empty\else%
    \PY@tok{#1}\expandafter\PY@toks\fi}
\def\PY@do#1{\PY@bc{\PY@tc{\PY@ul{%
    \PY@it{\PY@bf{\PY@ff{#1}}}}}}}
\def\PY#1#2{\PY@reset\PY@toks#1+\relax+\PY@do{#2}}

\expandafter\def\csname PY@tok@w\endcsname{\def\PY@tc##1{\textcolor[rgb]{0.73,0.73,0.73}{##1}}}
\expandafter\def\csname PY@tok@c\endcsname{\let\PY@it=\textit\def\PY@tc##1{\textcolor[rgb]{0.25,0.50,0.50}{##1}}}
\expandafter\def\csname PY@tok@cp\endcsname{\def\PY@tc##1{\textcolor[rgb]{0.74,0.48,0.00}{##1}}}
\expandafter\def\csname PY@tok@k\endcsname{\let\PY@bf=\textbf\def\PY@tc##1{\textcolor[rgb]{0.00,0.50,0.00}{##1}}}
\expandafter\def\csname PY@tok@kp\endcsname{\def\PY@tc##1{\textcolor[rgb]{0.00,0.50,0.00}{##1}}}
\expandafter\def\csname PY@tok@kt\endcsname{\def\PY@tc##1{\textcolor[rgb]{0.69,0.00,0.25}{##1}}}
\expandafter\def\csname PY@tok@o\endcsname{\def\PY@tc##1{\textcolor[rgb]{0.40,0.40,0.40}{##1}}}
\expandafter\def\csname PY@tok@ow\endcsname{\let\PY@bf=\textbf\def\PY@tc##1{\textcolor[rgb]{0.67,0.13,1.00}{##1}}}
\expandafter\def\csname PY@tok@nb\endcsname{\def\PY@tc##1{\textcolor[rgb]{0.00,0.50,0.00}{##1}}}
\expandafter\def\csname PY@tok@nf\endcsname{\def\PY@tc##1{\textcolor[rgb]{0.00,0.00,1.00}{##1}}}
\expandafter\def\csname PY@tok@nc\endcsname{\let\PY@bf=\textbf\def\PY@tc##1{\textcolor[rgb]{0.00,0.00,1.00}{##1}}}
\expandafter\def\csname PY@tok@nn\endcsname{\let\PY@bf=\textbf\def\PY@tc##1{\textcolor[rgb]{0.00,0.00,1.00}{##1}}}
\expandafter\def\csname PY@tok@ne\endcsname{\let\PY@bf=\textbf\def\PY@tc##1{\textcolor[rgb]{0.82,0.25,0.23}{##1}}}
\expandafter\def\csname PY@tok@nv\endcsname{\def\PY@tc##1{\textcolor[rgb]{0.10,0.09,0.49}{##1}}}
\expandafter\def\csname PY@tok@no\endcsname{\def\PY@tc##1{\textcolor[rgb]{0.53,0.00,0.00}{##1}}}
\expandafter\def\csname PY@tok@nl\endcsname{\def\PY@tc##1{\textcolor[rgb]{0.63,0.63,0.00}{##1}}}
\expandafter\def\csname PY@tok@ni\endcsname{\let\PY@bf=\textbf\def\PY@tc##1{\textcolor[rgb]{0.60,0.60,0.60}{##1}}}
\expandafter\def\csname PY@tok@na\endcsname{\def\PY@tc##1{\textcolor[rgb]{0.49,0.56,0.16}{##1}}}
\expandafter\def\csname PY@tok@nt\endcsname{\let\PY@bf=\textbf\def\PY@tc##1{\textcolor[rgb]{0.00,0.50,0.00}{##1}}}
\expandafter\def\csname PY@tok@nd\endcsname{\def\PY@tc##1{\textcolor[rgb]{0.67,0.13,1.00}{##1}}}
\expandafter\def\csname PY@tok@s\endcsname{\def\PY@tc##1{\textcolor[rgb]{0.73,0.13,0.13}{##1}}}
\expandafter\def\csname PY@tok@sd\endcsname{\let\PY@it=\textit\def\PY@tc##1{\textcolor[rgb]{0.73,0.13,0.13}{##1}}}
\expandafter\def\csname PY@tok@si\endcsname{\let\PY@bf=\textbf\def\PY@tc##1{\textcolor[rgb]{0.73,0.40,0.53}{##1}}}
\expandafter\def\csname PY@tok@se\endcsname{\let\PY@bf=\textbf\def\PY@tc##1{\textcolor[rgb]{0.73,0.40,0.13}{##1}}}
\expandafter\def\csname PY@tok@sr\endcsname{\def\PY@tc##1{\textcolor[rgb]{0.73,0.40,0.53}{##1}}}
\expandafter\def\csname PY@tok@ss\endcsname{\def\PY@tc##1{\textcolor[rgb]{0.10,0.09,0.49}{##1}}}
\expandafter\def\csname PY@tok@sx\endcsname{\def\PY@tc##1{\textcolor[rgb]{0.00,0.50,0.00}{##1}}}
\expandafter\def\csname PY@tok@m\endcsname{\def\PY@tc##1{\textcolor[rgb]{0.40,0.40,0.40}{##1}}}
\expandafter\def\csname PY@tok@gh\endcsname{\let\PY@bf=\textbf\def\PY@tc##1{\textcolor[rgb]{0.00,0.00,0.50}{##1}}}
\expandafter\def\csname PY@tok@gu\endcsname{\let\PY@bf=\textbf\def\PY@tc##1{\textcolor[rgb]{0.50,0.00,0.50}{##1}}}
\expandafter\def\csname PY@tok@gd\endcsname{\def\PY@tc##1{\textcolor[rgb]{0.63,0.00,0.00}{##1}}}
\expandafter\def\csname PY@tok@gi\endcsname{\def\PY@tc##1{\textcolor[rgb]{0.00,0.63,0.00}{##1}}}
\expandafter\def\csname PY@tok@gr\endcsname{\def\PY@tc##1{\textcolor[rgb]{1.00,0.00,0.00}{##1}}}
\expandafter\def\csname PY@tok@ge\endcsname{\let\PY@it=\textit}
\expandafter\def\csname PY@tok@gs\endcsname{\let\PY@bf=\textbf}
\expandafter\def\csname PY@tok@gp\endcsname{\let\PY@bf=\textbf\def\PY@tc##1{\textcolor[rgb]{0.00,0.00,0.50}{##1}}}
\expandafter\def\csname PY@tok@go\endcsname{\def\PY@tc##1{\textcolor[rgb]{0.53,0.53,0.53}{##1}}}
\expandafter\def\csname PY@tok@gt\endcsname{\def\PY@tc##1{\textcolor[rgb]{0.00,0.27,0.87}{##1}}}
\expandafter\def\csname PY@tok@err\endcsname{\def\PY@bc##1{\setlength{\fboxsep}{0pt}\fcolorbox[rgb]{1.00,0.00,0.00}{1,1,1}{\strut ##1}}}
\expandafter\def\csname PY@tok@kc\endcsname{\let\PY@bf=\textbf\def\PY@tc##1{\textcolor[rgb]{0.00,0.50,0.00}{##1}}}
\expandafter\def\csname PY@tok@kd\endcsname{\let\PY@bf=\textbf\def\PY@tc##1{\textcolor[rgb]{0.00,0.50,0.00}{##1}}}
\expandafter\def\csname PY@tok@kn\endcsname{\let\PY@bf=\textbf\def\PY@tc##1{\textcolor[rgb]{0.00,0.50,0.00}{##1}}}
\expandafter\def\csname PY@tok@kr\endcsname{\let\PY@bf=\textbf\def\PY@tc##1{\textcolor[rgb]{0.00,0.50,0.00}{##1}}}
\expandafter\def\csname PY@tok@bp\endcsname{\def\PY@tc##1{\textcolor[rgb]{0.00,0.50,0.00}{##1}}}
\expandafter\def\csname PY@tok@fm\endcsname{\def\PY@tc##1{\textcolor[rgb]{0.00,0.00,1.00}{##1}}}
\expandafter\def\csname PY@tok@vc\endcsname{\def\PY@tc##1{\textcolor[rgb]{0.10,0.09,0.49}{##1}}}
\expandafter\def\csname PY@tok@vg\endcsname{\def\PY@tc##1{\textcolor[rgb]{0.10,0.09,0.49}{##1}}}
\expandafter\def\csname PY@tok@vi\endcsname{\def\PY@tc##1{\textcolor[rgb]{0.10,0.09,0.49}{##1}}}
\expandafter\def\csname PY@tok@vm\endcsname{\def\PY@tc##1{\textcolor[rgb]{0.10,0.09,0.49}{##1}}}
\expandafter\def\csname PY@tok@sa\endcsname{\def\PY@tc##1{\textcolor[rgb]{0.73,0.13,0.13}{##1}}}
\expandafter\def\csname PY@tok@sb\endcsname{\def\PY@tc##1{\textcolor[rgb]{0.73,0.13,0.13}{##1}}}
\expandafter\def\csname PY@tok@sc\endcsname{\def\PY@tc##1{\textcolor[rgb]{0.73,0.13,0.13}{##1}}}
\expandafter\def\csname PY@tok@dl\endcsname{\def\PY@tc##1{\textcolor[rgb]{0.73,0.13,0.13}{##1}}}
\expandafter\def\csname PY@tok@s2\endcsname{\def\PY@tc##1{\textcolor[rgb]{0.73,0.13,0.13}{##1}}}
\expandafter\def\csname PY@tok@sh\endcsname{\def\PY@tc##1{\textcolor[rgb]{0.73,0.13,0.13}{##1}}}
\expandafter\def\csname PY@tok@s1\endcsname{\def\PY@tc##1{\textcolor[rgb]{0.73,0.13,0.13}{##1}}}
\expandafter\def\csname PY@tok@mb\endcsname{\def\PY@tc##1{\textcolor[rgb]{0.40,0.40,0.40}{##1}}}
\expandafter\def\csname PY@tok@mf\endcsname{\def\PY@tc##1{\textcolor[rgb]{0.40,0.40,0.40}{##1}}}
\expandafter\def\csname PY@tok@mh\endcsname{\def\PY@tc##1{\textcolor[rgb]{0.40,0.40,0.40}{##1}}}
\expandafter\def\csname PY@tok@mi\endcsname{\def\PY@tc##1{\textcolor[rgb]{0.40,0.40,0.40}{##1}}}
\expandafter\def\csname PY@tok@il\endcsname{\def\PY@tc##1{\textcolor[rgb]{0.40,0.40,0.40}{##1}}}
\expandafter\def\csname PY@tok@mo\endcsname{\def\PY@tc##1{\textcolor[rgb]{0.40,0.40,0.40}{##1}}}
\expandafter\def\csname PY@tok@ch\endcsname{\let\PY@it=\textit\def\PY@tc##1{\textcolor[rgb]{0.25,0.50,0.50}{##1}}}
\expandafter\def\csname PY@tok@cm\endcsname{\let\PY@it=\textit\def\PY@tc##1{\textcolor[rgb]{0.25,0.50,0.50}{##1}}}
\expandafter\def\csname PY@tok@cpf\endcsname{\let\PY@it=\textit\def\PY@tc##1{\textcolor[rgb]{0.25,0.50,0.50}{##1}}}
\expandafter\def\csname PY@tok@c1\endcsname{\let\PY@it=\textit\def\PY@tc##1{\textcolor[rgb]{0.25,0.50,0.50}{##1}}}
\expandafter\def\csname PY@tok@cs\endcsname{\let\PY@it=\textit\def\PY@tc##1{\textcolor[rgb]{0.25,0.50,0.50}{##1}}}

\def\PYZbs{\char`\\}
\def\PYZus{\char`\_}
\def\PYZob{\char`\{}
\def\PYZcb{\char`\}}
\def\PYZca{\char`\^}
\def\PYZam{\char`\&}
\def\PYZlt{\char`\<}
\def\PYZgt{\char`\>}
\def\PYZsh{\char`\#}
\def\PYZpc{\char`\%}
\def\PYZdl{\char`\$}
\def\PYZhy{\char`\-}
\def\PYZsq{\char`\'}
\def\PYZdq{\char`\"}
\def\PYZti{\char`\~}
% for compatibility with earlier versions
\def\PYZat{@}
\def\PYZlb{[}
\def\PYZrb{]}
\makeatother


    % Exact colors from NB
    \definecolor{incolor}{rgb}{0.0, 0.0, 0.5}
    \definecolor{outcolor}{rgb}{0.545, 0.0, 0.0}



    
    % Prevent overflowing lines due to hard-to-break entities
    \sloppy 
    % Setup hyperref package
    \hypersetup{
      breaklinks=true,  % so long urls are correctly broken across lines
      colorlinks=true,
      urlcolor=urlcolor,
      linkcolor=linkcolor,
      citecolor=citecolor,
      }
    % Slightly bigger margins than the latex defaults
    
    \geometry{verbose,tmargin=1in,bmargin=1in,lmargin=1in,rmargin=1in}
    
    

    \begin{document}
    
    
    \maketitle
    
    

    
    \hypertarget{eecs-491---a2---e1}{%
\section{EECS 491 - A2 - E1}\label{eecs-491---a2---e1}}

\hypertarget{tristan-maidment---tdm47}{%
\subsubsection{Tristan Maidment -
tdm47}\label{tristan-maidment---tdm47}}

\hypertarget{goal}{%
\subsection{Goal}\label{goal}}

The goal for this exercise is to use the TCP data acquired from the
wireshark dump made in A1. By using the same model, we can create a
simple example with real data that can be solved via sampling and
variable elimination.

\hypertarget{implementation}{%
\subsection{Implementation}\label{implementation}}

    \begin{Verbatim}[commandchars=\\\{\}]
{\color{incolor}In [{\color{incolor}1}]:} \PY{k+kn}{import} \PY{n+nn}{pymc3} \PY{k}{as} \PY{n+nn}{pm}
        \PY{k+kn}{import} \PY{n+nn}{csv}
        \PY{k+kn}{import} \PY{n+nn}{matplotlib}\PY{n+nn}{.}\PY{n+nn}{pyplot} \PY{k}{as} \PY{n+nn}{plt}
        \PY{k+kn}{import} \PY{n+nn}{numpy} \PY{k}{as} \PY{n+nn}{np}
        \PY{k+kn}{import} \PY{n+nn}{scipy}\PY{n+nn}{.}\PY{n+nn}{stats} \PY{k}{as} \PY{n+nn}{stats}
        \PY{k+kn}{import} \PY{n+nn}{math}
        \PY{k+kn}{from} \PY{n+nn}{pymc3}\PY{n+nn}{.}\PY{n+nn}{math} \PY{k}{import} \PY{n}{switch}
        \PY{k+kn}{from} \PY{n+nn}{pymc3} \PY{k}{import} \PY{n}{Bernoulli}
        \PY{o}{\PYZpc{}}\PY{k}{matplotlib} inline
\end{Verbatim}


    \begin{Verbatim}[commandchars=\\\{\}]
/usr/local/lib/python3.6/site-packages/h5py/\_\_init\_\_.py:36: FutureWarning: Conversion of the second argument of issubdtype from `float` to `np.floating` is deprecated. In future, it will be treated as `np.float64 == np.dtype(float).type`.
  from .\_conv import register\_converters as \_register\_converters

    \end{Verbatim}

    The data from A1 is imported, and a new parameter,
\texttt{hops\_retrans\_all}, is used to determine all retransmission
events, including ones that are not Fast Retransmits and Timeouts.

    \begin{Verbatim}[commandchars=\\\{\}]
{\color{incolor}In [{\color{incolor}2}]:} \PY{n}{hops} \PY{o}{=} \PY{p}{[}\PY{p}{]}
        \PY{n}{hops\PYZus{}retrans} \PY{o}{=} \PY{p}{[}\PY{p}{]}
        \PY{n}{hops\PYZus{}timeout} \PY{o}{=} \PY{p}{[}\PY{p}{]}
        \PY{n}{hops\PYZus{}retrans\PYZus{}all} \PY{o}{=} \PY{p}{[}\PY{p}{]}
        \PY{k}{with} \PY{n+nb}{open}\PY{p}{(}\PY{l+s+s1}{\PYZsq{}}\PY{l+s+s1}{TTL3.csv}\PY{l+s+s1}{\PYZsq{}}\PY{p}{,} \PY{n}{newline}\PY{o}{=}\PY{l+s+s1}{\PYZsq{}}\PY{l+s+s1}{\PYZsq{}}\PY{p}{)} \PY{k}{as} \PY{n}{csvfile}\PY{p}{:}
            \PY{n}{ttls} \PY{o}{=} \PY{n}{csv}\PY{o}{.}\PY{n}{reader}\PY{p}{(}\PY{n}{csvfile}\PY{p}{,} \PY{n}{delimiter}\PY{o}{=}\PY{l+s+s1}{\PYZsq{}}\PY{l+s+s1}{,}\PY{l+s+s1}{\PYZsq{}}\PY{p}{,} \PY{n}{quotechar}\PY{o}{=}\PY{l+s+s1}{\PYZsq{}}\PY{l+s+s1}{\PYZdq{}}\PY{l+s+s1}{\PYZsq{}}\PY{p}{)}
            \PY{k}{for} \PY{n}{row} \PY{o+ow}{in} \PY{n}{ttls}\PY{p}{:}
                \PY{k}{if} \PY{n}{row}\PY{p}{[}\PY{l+m+mi}{1}\PY{p}{]}\PY{p}{:}
                    \PY{n}{ttl} \PY{o}{=} \PY{l+m+mi}{64} \PY{o}{\PYZhy{}} \PY{n+nb}{int}\PY{p}{(}\PY{n}{row}\PY{p}{[}\PY{l+m+mi}{1}\PY{p}{]}\PY{p}{)}
                    \PY{c+c1}{\PYZsh{}I removed all TTL that are below 0.  }
                    \PY{c+c1}{\PYZsh{}These TTL are either set at a value higher than 64 by the sender,}
                    \PY{c+c1}{\PYZsh{}or packets being sent to local devices on my home network. (TTL = 0)}
                    \PY{k}{if} \PY{n}{ttl} \PY{o}{\PYZgt{}} \PY{l+m+mi}{0}\PY{p}{:}
                        \PY{k}{if} \PY{l+s+s2}{\PYZdq{}}\PY{l+s+s2}{[TCP Retransmission]}\PY{l+s+s2}{\PYZdq{}} \PY{o+ow}{in} \PY{n}{row}\PY{p}{[}\PY{l+m+mi}{2}\PY{p}{]}\PY{p}{:}
                            \PY{n}{hops\PYZus{}timeout}\PY{o}{.}\PY{n}{append}\PY{p}{(}\PY{n}{ttl}\PY{p}{)}
                        \PY{k}{elif} \PY{l+s+s2}{\PYZdq{}}\PY{l+s+s2}{[TCP Fast Retransmission]}\PY{l+s+s2}{\PYZdq{}} \PY{o+ow}{in} \PY{n}{row}\PY{p}{[}\PY{l+m+mi}{2}\PY{p}{]}\PY{p}{:}
                            \PY{n}{hops\PYZus{}retrans}\PY{o}{.}\PY{n}{append}\PY{p}{(}\PY{n}{ttl}\PY{p}{)}
                        \PY{k}{elif} \PY{l+s+s2}{\PYZdq{}}\PY{l+s+s2}{retransmission}\PY{l+s+s2}{\PYZdq{}} \PY{o+ow}{in} \PY{n}{row}\PY{p}{[}\PY{l+m+mi}{2}\PY{p}{]}\PY{p}{:}
                            \PY{n}{hops\PYZus{}retrans\PYZus{}all}\PY{o}{.}\PY{n}{append}\PY{p}{(}\PY{n}{ttl}\PY{p}{)}
                        \PY{k}{else}\PY{p}{:}
                            \PY{n}{hops}\PY{o}{.}\PY{n}{append}\PY{p}{(}\PY{n}{ttl}\PY{p}{)}
        \PY{n}{pkt\PYZus{}total} \PY{o}{=} \PY{n+nb}{len}\PY{p}{(}\PY{n}{hops}\PY{p}{)} \PY{o}{+} \PY{n+nb}{len}\PY{p}{(}\PY{n}{hops\PYZus{}timeout}\PY{p}{)} \PY{o}{+} \PY{n+nb}{len}\PY{p}{(}\PY{n}{hops\PYZus{}retrans}\PY{p}{)} \PY{o}{+} \PY{n+nb}{len}\PY{p}{(}\PY{n}{hops\PYZus{}retrans\PYZus{}all}\PY{p}{)}
        \PY{n}{pkt\PYZus{}succ} \PY{o}{=} \PY{n+nb}{len}\PY{p}{(}\PY{n}{hops}\PY{p}{)}
        \PY{n}{pkt\PYZus{}time} \PY{o}{=} \PY{n+nb}{len}\PY{p}{(}\PY{n}{hops\PYZus{}timeout}\PY{p}{)}
        \PY{n}{pkt\PYZus{}fast} \PY{o}{=} \PY{n+nb}{len}\PY{p}{(}\PY{n}{hops\PYZus{}retrans}\PY{p}{)}
        \PY{n+nb}{print}\PY{p}{(}\PY{l+s+s2}{\PYZdq{}}\PY{l+s+s2}{Packets successful}\PY{l+s+s2}{\PYZdq{}}\PY{p}{,} \PY{n}{pkt\PYZus{}succ}\PY{p}{)}
        \PY{n+nb}{print}\PY{p}{(}\PY{l+s+s2}{\PYZdq{}}\PY{l+s+s2}{Packets unsuccessful}\PY{l+s+s2}{\PYZdq{}}\PY{p}{,} \PY{n}{pkt\PYZus{}total} \PY{o}{\PYZhy{}} \PY{n}{pkt\PYZus{}succ}\PY{p}{)}
        \PY{n+nb}{print}\PY{p}{(}\PY{l+s+s2}{\PYZdq{}}\PY{l+s+s2}{Packets timed\PYZus{}out}\PY{l+s+s2}{\PYZdq{}}\PY{p}{,} \PY{n}{pkt\PYZus{}time}\PY{p}{)}
        \PY{n+nb}{print}\PY{p}{(}\PY{l+s+s2}{\PYZdq{}}\PY{l+s+s2}{Packets fast\PYZus{}retransmitted}\PY{l+s+s2}{\PYZdq{}}\PY{p}{,} \PY{n}{pkt\PYZus{}fast}\PY{p}{)}
        \PY{n+nb}{print}\PY{p}{(}\PY{l+s+s2}{\PYZdq{}}\PY{l+s+s2}{Percent unsuccessful}\PY{l+s+s2}{\PYZdq{}}\PY{p}{,} \PY{p}{(}\PY{n}{pkt\PYZus{}total} \PY{o}{\PYZhy{}} \PY{n}{pkt\PYZus{}succ}\PY{p}{)}\PY{o}{/}\PY{n}{pkt\PYZus{}total}\PY{p}{)}
        \PY{n+nb}{print}\PY{p}{(}\PY{l+s+s2}{\PYZdq{}}\PY{l+s+s2}{Percent timed\PYZus{}out}\PY{l+s+s2}{\PYZdq{}}\PY{p}{,} \PY{n}{pkt\PYZus{}time}\PY{o}{/}\PY{n}{pkt\PYZus{}total}\PY{p}{)}
        \PY{n+nb}{print}\PY{p}{(}\PY{l+s+s2}{\PYZdq{}}\PY{l+s+s2}{Percent fast\PYZus{}retransmitted}\PY{l+s+s2}{\PYZdq{}}\PY{p}{,} \PY{n}{pkt\PYZus{}fast}\PY{o}{/}\PY{n}{pkt\PYZus{}total}\PY{p}{)}
        \PY{n+nb}{print}\PY{p}{(}\PY{l+s+s2}{\PYZdq{}}\PY{l+s+s2}{Packets total}\PY{l+s+s2}{\PYZdq{}}\PY{p}{,} \PY{n}{pkt\PYZus{}total}\PY{p}{)}
\end{Verbatim}


    \begin{Verbatim}[commandchars=\\\{\}]
Packets successful 104458
Packets unsuccessful 204
Packets timed\_out 73
Packets fast\_retransmitted 43
Percent unsuccessful 0.0019491314899390418
Percent timed\_out 0.0006974833272821081
Percent fast\_retransmitted 0.0004108463434675431
Packets total 104662

    \end{Verbatim}

    \[
\begin{align}
P(R) & = 0.0019491314899390418 \\
P(T | R) & = 0.357843137254902 \\ 
P(F | R) & = 0.21078431372549 \\
P(T | \bar R) & = 0.0111083296707496513 \\ 
P(F | \bar R) & = 0.0236974833272821081 \\
P(T) &= 0.0006974833272821081 \\
P(F) &= 0.0004108463434675431 \\
\end{align}
\]

Due to inaccuracies present in Wireshark, and the present of other
modern TCP optimizations, we will assume that Wiresharks representation
of P(T) and P(F) are independent. For the purpose of this exercise, we
will assume that we estimated the following:

\[
\begin{align}
P(T = true| R = false, F = false) & = 0.3212359034102345301 \\
P(T = true| R = false, F = true) & = 0.0111083296707496513 \\
P(T = true| R = true, F = false) & = 0.1767348075841529761 \\
P(T = true| R = true, F = true) & = 0.0236974833272821081 \\
\end{align}
\]

    Using this data we can define the model.

    \begin{Verbatim}[commandchars=\\\{\}]
{\color{incolor}In [{\color{incolor}32}]:} \PY{k}{with} \PY{n}{pm}\PY{o}{.}\PY{n}{Model}\PY{p}{(}\PY{p}{)} \PY{k}{as} \PY{n}{model}\PY{p}{:}
             \PY{n}{retransmit} \PY{o}{=} \PY{n}{Bernoulli}\PY{p}{(}\PY{l+s+s1}{\PYZsq{}}\PY{l+s+s1}{Retransmit}\PY{l+s+s1}{\PYZsq{}}\PY{p}{,} \PY{l+m+mf}{0.0019491314899390418}\PY{p}{)}
             \PY{n}{fast\PYZus{}retransmit} \PY{o}{=} \PY{n}{Bernoulli}\PY{p}{(}\PY{l+s+s1}{\PYZsq{}}\PY{l+s+s1}{Fast Retransmit}\PY{l+s+s1}{\PYZsq{}}\PY{p}{,} \PY{n}{switch}\PY{p}{(}\PY{n}{retransmit}\PY{p}{,} \PY{l+m+mf}{0.21078431372549}\PY{p}{,} \PY{l+m+mf}{0.0236974833272821081}\PY{p}{)}\PY{p}{)}
             \PY{n}{timeout} \PY{o}{=} \PY{n}{Bernoulli}\PY{p}{(}\PY{l+s+s1}{\PYZsq{}}\PY{l+s+s1}{Timeout}\PY{l+s+s1}{\PYZsq{}}\PY{p}{,} \PY{n}{switch}\PY{p}{(}\PY{n}{retransmit}\PY{p}{,} \PY{l+m+mf}{0.357843137254902}\PY{p}{,} \PY{l+m+mf}{0.0111083296707496513}\PY{p}{)}\PY{p}{)}
\end{Verbatim}


    Using that model, we can sample the network. Due to the very low
probabilities, a large sampling value must be used for accurate results.

    \begin{Verbatim}[commandchars=\\\{\}]
{\color{incolor}In [{\color{incolor}33}]:} \PY{k}{with} \PY{n}{model}\PY{p}{:}
                 \PY{n}{trace} \PY{o}{=} \PY{n}{pm}\PY{o}{.}\PY{n}{sample}\PY{p}{(}\PY{l+m+mi}{40000}\PY{p}{,} \PY{n}{chains}\PY{o}{=}\PY{l+m+mi}{2}\PY{p}{)}
                 \PY{n}{pm}\PY{o}{.}\PY{n}{traceplot}\PY{p}{(}\PY{n}{trace}\PY{p}{)}
\end{Verbatim}


    \begin{Verbatim}[commandchars=\\\{\}]
Multiprocess sampling (2 chains in 2 jobs)
BinaryGibbsMetropolis: [Retransmit, Fast Retransmit, Timeout]
100\%|██████████| 40500/40500 [00:15<00:00, 2655.16it/s]

    \end{Verbatim}

    \begin{center}
    \adjustimage{max size={0.9\linewidth}{0.9\paperheight}}{output_8_1.png}
    \end{center}
    { \hspace*{\fill} \\}
    
    Imported from the ``Wet Grass'' example to determine the conditional
probabilities from the trace.

    \begin{Verbatim}[commandchars=\\\{\}]
{\color{incolor}In [{\color{incolor}34}]:} \PY{c+c1}{\PYZsh{}\PYZsh{} From \PYZdq{}Wet Grass Example\PYZdq{}}
         \PY{k}{def} \PY{n+nf}{calcCondProb}\PY{p}{(}\PY{n}{trace}\PY{p}{,} \PY{n}{event}\PY{p}{,} \PY{n}{cond}\PY{p}{)}\PY{p}{:}
             \PY{c+c1}{\PYZsh{} find all samples satisfy conditions}
             \PY{k}{for} \PY{n}{k}\PY{p}{,} \PY{n}{v} \PY{o+ow}{in} \PY{n}{cond}\PY{o}{.}\PY{n}{items}\PY{p}{(}\PY{p}{)}\PY{p}{:}
                 \PY{n}{trace} \PY{o}{=} \PY{p}{[}\PY{n}{smp} \PY{k}{for} \PY{n}{smp} \PY{o+ow}{in} \PY{n}{trace} \PY{k}{if} \PY{n}{smp}\PY{p}{[}\PY{n}{k}\PY{p}{]} \PY{o}{==} \PY{n}{v}\PY{p}{]}
             \PY{c+c1}{\PYZsh{} record quantity of all samples fulfill condition}
             \PY{n}{nCondSample} \PY{o}{=} \PY{n+nb}{len}\PY{p}{(}\PY{n}{trace}\PY{p}{)}
             \PY{c+c1}{\PYZsh{} find all samples satisfy event}
             \PY{k}{for} \PY{n}{k}\PY{p}{,} \PY{n}{v} \PY{o+ow}{in} \PY{n}{event}\PY{o}{.}\PY{n}{items}\PY{p}{(}\PY{p}{)}\PY{p}{:}
                 \PY{n}{trace} \PY{o}{=} \PY{p}{[}\PY{n}{smp} \PY{k}{for} \PY{n}{smp} \PY{o+ow}{in} \PY{n}{trace} \PY{k}{if} \PY{n}{smp}\PY{p}{[}\PY{n}{k}\PY{p}{]} \PY{o}{==} \PY{n}{v}\PY{p}{]}
             \PY{c+c1}{\PYZsh{} calculate conditional probability}
             \PY{k}{return} \PY{n+nb}{len}\PY{p}{(}\PY{n}{trace}\PY{p}{)} \PY{o}{/} \PY{n}{nCondSample}
\end{Verbatim}


    \begin{Verbatim}[commandchars=\\\{\}]
{\color{incolor}In [{\color{incolor}35}]:} \PY{n+nb}{print}\PY{p}{(}\PY{l+s+s1}{\PYZsq{}}\PY{l+s+s1}{P(R = true | T = true) = }\PY{l+s+s1}{\PYZsq{}}\PY{p}{,} 
               \PY{n}{calcCondProb}\PY{p}{(}\PY{n}{trace}\PY{p}{,} \PY{p}{\PYZob{}}\PY{l+s+s1}{\PYZsq{}}\PY{l+s+s1}{Retransmit}\PY{l+s+s1}{\PYZsq{}} \PY{p}{:} \PY{l+m+mi}{1}\PY{p}{\PYZcb{}}\PY{p}{,} \PY{p}{\PYZob{}}\PY{l+s+s1}{\PYZsq{}}\PY{l+s+s1}{Timeout}\PY{l+s+s1}{\PYZsq{}} \PY{p}{:} \PY{l+m+mi}{1}\PY{p}{\PYZcb{}}\PY{p}{)}\PY{p}{)}
         \PY{n+nb}{print}\PY{p}{(}\PY{l+s+s1}{\PYZsq{}}\PY{l+s+s1}{P(R = true | T = false) = }\PY{l+s+s1}{\PYZsq{}}\PY{p}{,} 
               \PY{n}{calcCondProb}\PY{p}{(}\PY{n}{trace}\PY{p}{,} \PY{p}{\PYZob{}}\PY{l+s+s1}{\PYZsq{}}\PY{l+s+s1}{Retransmit}\PY{l+s+s1}{\PYZsq{}} \PY{p}{:} \PY{l+m+mi}{1}\PY{p}{\PYZcb{}}\PY{p}{,} \PY{p}{\PYZob{}}\PY{l+s+s1}{\PYZsq{}}\PY{l+s+s1}{Timeout}\PY{l+s+s1}{\PYZsq{}} \PY{p}{:} \PY{l+m+mi}{0}\PY{p}{\PYZcb{}}\PY{p}{)}\PY{p}{)}
         \PY{n+nb}{print}\PY{p}{(}\PY{l+s+s1}{\PYZsq{}}\PY{l+s+s1}{P(R = true | F = true) = }\PY{l+s+s1}{\PYZsq{}}\PY{p}{,} 
               \PY{n}{calcCondProb}\PY{p}{(}\PY{n}{trace}\PY{p}{,} \PY{p}{\PYZob{}}\PY{l+s+s1}{\PYZsq{}}\PY{l+s+s1}{Retransmit}\PY{l+s+s1}{\PYZsq{}} \PY{p}{:} \PY{l+m+mi}{1}\PY{p}{\PYZcb{}}\PY{p}{,} \PY{p}{\PYZob{}}\PY{l+s+s1}{\PYZsq{}}\PY{l+s+s1}{Fast Retransmit}\PY{l+s+s1}{\PYZsq{}} \PY{p}{:} \PY{l+m+mi}{1}\PY{p}{\PYZcb{}}\PY{p}{)}\PY{p}{)}
         \PY{n+nb}{print}\PY{p}{(}\PY{l+s+s1}{\PYZsq{}}\PY{l+s+s1}{P(R = true | F = false) = }\PY{l+s+s1}{\PYZsq{}}\PY{p}{,} 
               \PY{n}{calcCondProb}\PY{p}{(}\PY{n}{trace}\PY{p}{,} \PY{p}{\PYZob{}}\PY{l+s+s1}{\PYZsq{}}\PY{l+s+s1}{Retransmit}\PY{l+s+s1}{\PYZsq{}} \PY{p}{:} \PY{l+m+mi}{1}\PY{p}{\PYZcb{}}\PY{p}{,} \PY{p}{\PYZob{}}\PY{l+s+s1}{\PYZsq{}}\PY{l+s+s1}{Fast Retransmit}\PY{l+s+s1}{\PYZsq{}} \PY{p}{:} \PY{l+m+mi}{0}\PY{p}{\PYZcb{}}\PY{p}{)}\PY{p}{)}
\end{Verbatim}


    \begin{Verbatim}[commandchars=\\\{\}]
P(R = true | T = true) =  0.06595744680851064
P(R = true | T = false) =  0.0010624841892233746
P(R = true | F = true) =  0.017130620985010708
P(R = true | F = false) =  0.0014590692673936416

    \end{Verbatim}

    We now know \texttt{P(R\ =\ true\ \textbar{}\ T\ =\ true)}. Due to the
small probabilities, the value of this fluctuates betwee 0.04 - 0.07.
When calculating the conditional probability using variable elimination,
we get the value \textasciitilde{}0.05. This seems to be accurate to the
sampling result.

    \hypertarget{variable-elimination}{%
\subsubsection{Variable Elimination}\label{variable-elimination}}

\[
\begin{align}
P(R = true | T = true) & = \sum_{F} P(R, F|T) \\
& = \sum_{F} \dfrac{P(R,F,T)}{P(T)} \\
& = \dfrac{P(R) \sum_{F} P(T|R,F)P(F)}{\sum_{R} P(R) \sum_{F} P(T|R,F) P(F)} \\
P(R) \sum_{F} P(T|R,F)P(F) & = 0.0019491314899390418 \times (0.176734807584152976 \times (1-0.0004108463434675431) + 0.0236974833272821081 \times 0.0004108463434675431) \approx 0.00034435682 \\
\dfrac{P(R) \sum_{F} P(T|R,F)P(F)}{\sum_{R} P(R) \sum_{F} P(T|R,F) P(F)} & = \dfrac{0.00034435682}{0.00034435682 + 0.00619382033} \\
& \approx 0.05266862798
\end{align}
\]

    \hypertarget{conclusion}{%
\subsection{Conclusion}\label{conclusion}}

From this exercise, it can be concluded that the sampling method
provides an accurate alternative for variable elimination, which is
computationally expensive and slow to do by hand. In addition, it can be
noted that networks with very small probabilities have a large amount of
variance.


    % Add a bibliography block to the postdoc
    
    
    
    \end{document}
